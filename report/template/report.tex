\documentclass[a4paper, 11pt]{article}

\usepackage[ngerman]{babel}
\usepackage[T1]{fontenc}
\usepackage[utf8]{inputenc}
\usepackage{textcomp}

\usepackage{amsmath,amssymb}
\usepackage{lmodern}
\usepackage{iftex}
\usepackage[
    top=1.5cm,
    bottom=0.6cm,
    left=1.5cm,
    right=1.5cm
]{geometry}
\usepackage{xcolor}
\usepackage{longtable,booktabs,array}
\usepackage{calc}
\usepackage{etoolbox}
\usepackage{footnote}
\usepackage{graphicx}
\usepackage{hyperref}

\PassOptionsToPackage{unicode}{hyperref}
\PassOptionsToPackage{hyphens}{url}

\makeatletter
\@ifundefined{KOMAClassName}{% if non-KOMA class
    \IfFileExists{parskip.sty}{%
        \usepackage{parskip}
        \usepackage{wasysym}
    }{% else
        \setlength{\parindent}{0pt}
        \setlength{\parskip}{6pt plus 2pt minus 1pt}}
}{% if KOMA class
    \KOMAoptions{parskip=half}}
\makeatother

\makeatletter
\patchcmd\longtable{\par}{\if@noskipsec\mbox{}\fi\par}{}{}
\makeatother

\makesavenoteenv{longtable}
\makeatletter
\def\maxwidth{\ifdim\Gin@nat@width>\linewidth\linewidth\else\Gin@nat@width\fi}
\def\maxheight{\ifdim\Gin@nat@height>\textheight\textheight\else\Gin@nat@height\fi}
\makeatother

\setkeys{Gin}{width=\maxwidth,height=\maxheight,keepaspectratio}

\makeatletter
\def\fps@figure{htbp}
\makeatother

\setlength{\emergencystretch}{3em}
\providecommand{\tightlist}{%
    \setlength{\itemsep}{0pt}\setlength{\parskip}{0pt}}
\setcounter{secnumdepth}{-\maxdimen}
\urlstyle{same}
\hypersetup{
    hidelinks,
    pdfcreator={LaTeX via pandoc}}

\author{}
\date{}

\begin{document}
    \pagestyle{empty}

    % \hfill
    % \includegraphics[height=0.75972in]{logo}
    % \\
    \Large\textbf{Sonderbericht zu besonderen Vorkommnissen}\normalsize

    \begin{longtable}[]{@{}
        |>{\raggedright\arraybackslash}p{(\columnwidth - 0\tabcolsep) * \real{1.0000}}|@{}}
        \hline
        \endhead
        \begin{center}
            beim Spiel der {{ .Header.Team }} - Mannschaften

                {{ .Header.Home }} - {{ .Header.Away }}

            am {{ .Header.Date }} auf dem Sportplatz in {{ .Header.Place }}.

            Halbzeitstand: {{ .Header.HomeScoreHalf }} - {{ .Header.AwayScoreHalf }} Endstand: {{ .Header.HomeScore }} - {{ .Header.AwayScore }} Spielkennung: {{ .Header.MatchIdentifier }}

            Spielklasse:
                {{ if .Header.MatchClass.Federation }} \checkmark{} {{ else }} $\Box$ {{ end }} Verbandsspiel
                {{ if .Header.MatchClass.Private }} \checkmark{} {{ else }} $\Box$ {{ end }} Privatspiel
                {{ if .Header.MatchClass.Other }} \checkmark{} {{ else }} $\Box$ {{ end }} Sonstiges Spiel
        \end{center}
        \\
        \hline
    \end{longtable}

    \def\arraystretch{1.2}
    \begin{longtable}[]{@{}
        |>{\raggedright\arraybackslash}p{(\columnwidth - 2\tabcolsep) * \real{0.37}}
        |>{\raggedright\arraybackslash}p{(\columnwidth - 2\tabcolsep) * \real{0.63}}|@{}}
        \hline
        \endhead
        \textbf{Betreff} & \textbf{ {{ .Content.Subject }} }

        \textbf{Passnummer: {{ .Content.PassportId }}} \\\hline
        \textbf{Wer} beging das Vergehen? (z.B. Spieler, Trainer, Zuschauer usw. und dessen Verein) & {{ .Content.Who }} \\\hline
        \textbf{Wann} ereignete sich das Vergehen? (Spielminute)                                                                 & {{ .Content.When }}          \\\hline
        \textbf{Wie} war der Spielstand?                                                                                         & {{ .Content.How }}           \\\hline
        \textbf{Was}\footnote{allgemein gehaltene Begriffe (beleidigte,
            beschimpfte) sind nicht zu verwenden. Es muss konkret angegeben werden
            was z.B. gesagt wurde. Gleichwohl sind Absicht oder andere nicht
            messbare Begriffe zu vermeiden, genauso wie Begriffe, die in
            Sportgerichtsurteilen vorkommen (z.B. rohes Spiel, Tätlichkeit)}
        war das Vergehen? (genaue Beschreibung was die fehlbare Person gemacht hat)                                              & {{ .Content.What }}          \\\hline
        \textbf{Gegen wen} ging das Vergehen? (z.B. Gegenspieler, Mitspieler, Zuschauer usw.) & {{ .Content.Against }} \\\hline
        \textbf{Wo} war das Vergehen? (genauer Ort des Vergehens)                                                                & {{ .Content.Where }}         \\\hline
        \textbf{Wo} war der Ball beim Vergehen?                                                                                  & {{ .Content.WhereBall }}     \\\hline
        \textbf{Wo} stand der SR bzw. der SRA beim Vorfall?                                                                      & {{ .Content.WhereReferee }}  \\\hline
        \textbf{War} der Spieler bereits verwarnt?                                                                               & {{ .Content.AlreadyWarned }} \\\hline
        \textbf{Wurde} der fehlbare Spieler vorher provoziert oder gefoult?                                                      & {{ .Content.WasProvoked }}   \\\hline
        Konnte der gefoulte \textbf{Spieler weiterspielen} oder \textbf{musste er ausgewechselt werden}? & {{ .Content.CouldPlayOn }} \\\hline
        \textbf{Wie} und \textbf{wo} wurde das Spiel fortgesetzt?                                                                & {{ .Content.HowContinued }}  \\\hline
        \textbf{Welche} Wirkung wurde bei der Spielfortsetzung erzielt? (z.B. bei SST oder FST in Tornähe) & {{ .Content.WhatEffect }} \\\hline
        \textbf{Verhalten} des fehlbaren Spielers \textbf{nach} dem Feldverweis (auf dem Platz / in der Kabine / nach dem Spiel) & {{ .Content.BehaviorAfter }} \\\hline
        \textbf{Sonstiges} (z.B. witterungsbedingter Spielabbruch, Passrechtliches und dgl.) & {{ .Content.Other }} \\
        \hline
    \end{longtable}

    \begin{longtable}[]{@{}
        |>{\raggedright\arraybackslash}p{(\columnwidth - 2\tabcolsep) * \real{0.5000}}
        |>{\raggedright\arraybackslash}p{(\columnwidth - 2\tabcolsep) * \real{0.5000}}|@{}}
        \hline
        \endhead
        \begin{minipage}[t]{\linewidth}
            \raggedright
            {{ .Footer.Place }}, den \today \\
            \strut \\
            {{ .Footer.Name }}

            Schiedsrichter\strut
        \end{minipage} & \begin{minipage}[t]{\linewidth}
                             \raggedright
                             Kontaktdaten SR:\\
                             \emph{Straße: {{ .Footer.Street }} PLZ: {{ .Footer.Zip }} Ort: {{ .Footer.Place }}\\
                             Schiedsrichtergruppe: {{ .Footer.Group }}}

                             \emph{Angaben freigestellt:}

                             \emph{Tel.: {{ .Footer.Phone }} Email: {{ .Footer.Mail }}}\strut
        \end{minipage} \\
        \hline
        \begin{minipage}[t]{\linewidth}
            \raggedright
            Verteiler:

            \checkmark{} Original zum Spielbericht

                {{ if .Footer.Recipients.GSO }} \checkmark{} {{ else }} $\Box$ {{ end }} GSO
                {{ if .Footer.Recipients.KSO }} \checkmark{} {{ else }} $\Box$ {{ end }} KSO
                {{ if .Footer.Recipients.BSO }} \checkmark{} {{ else }} $\Box$ {{ end }} BSO
                {{ if .Footer.Recipients.VSO }} \checkmark{} {{ else }} $\Box$ {{ end }} VSO\\
            \checkmark{} SR -- eigene Unterlagen\strut
        \end{minipage} & \begin{minipage}[t]{\linewidth}
                             \raggedright
                             Durchschrift an den betroffenen Verein:

                             Versand über DFB-Net SpielPlus:

                             Sonstige Mailadresse:\\
                             \emph{ }\strut
        \end{minipage} \\
        \hline
    \end{longtable}

\end{document}
